\documentclass{article}
\usepackage{tikz}
\usepgfmodule{oo}
\begin{document}
\pgfooclass{stamp}{
    % This is the class stamp
    \method stamp() { 
        % The constructor
    }
    \method apply(#1,#2,#3) { % Causes the stamp to be shown at coordinate (#1,#2)
        % Draw the stamp:
        \node [rotate=20,font=\huge] at (#1,#2) {#3};
    }
}

\pgfooclass{demo}{
    % This is the class stamp
    \attribute text;
    \attribute rotation angle = 20;
    \method demo(#1) {
        \pgfooset{text}{#1} % Set the text
    }
    \method apply(#1,#2) {
        \pgfoothis.shift origin(#1,#2)
        % Draw the stamp:
        \node [rotate=\pgfoovalueof{rotation angle},font=\huge]{\pgfoovalueof{text}};
    }
    \method shift origin(#1,#2) { \tikzset{xshift=#1,yshift=#2} }
    \method set rotation (#1) {
    \pgfooset{rotation angle}{#1}
    }
}

\pgfoonew \mystamp=new stamp()
\begin{tikzpicture}
\mystamp.apply(1,2,Passed)
\mystamp.apply(3,4,Nice)
\end{tikzpicture}

% Create a new object:
\pgfoonew \mydemo=new demo()
\begin{tikzpicture}
% Get the object's id and store it in \myid:
\mydemo.get id(\myid)
% The following two calls have the same effect:
\mydemo.apply(1,1)
\pgfooobj{\myid}.apply(1,1)
\end{tikzpicture}
\end{document}