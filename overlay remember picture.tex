\documentclass{article}
\usepackage{tikz}
\begin{document}
Inside the current text we place two pictures, containing nodes named \texttt{n1} and \texttt{n2}, using

\verb|\tikz[remember picture] \node[circle,fill=red!50] (n1) {};|

which yields \tikz[remember picture] \node[circle,fill=red!50] (n1) {}; , and

\verb|\tikz[remember picture] \node[fill=blue!50] (n2) {};|

yielding the code \tikz[remember picture] \node[fill=blue!50] (n2) {};. To connect these codes, we create another picture using the overlay option and also the \texttt{remember picture} option.

\begin{verbatim}
\begin{tikzpicture}[remember picture,overlay]
\draw[->,very thick] (n1) -- (n2);
\end{tikzpicture}
\end{verbatim}

Note that the last picture is seemingly empty. What happens is that it has zero size and contains an arrow
that lies well outside its bounds. As a last example, we connect a node in another picture to the first two
nodes. Here, we provide the \texttt{overlay} option only with the line that we do not wish to count as part of the
picture.
\begin{tikzpicture}[overlay,remember picture]
\node (c) [circle,draw] {Big circle};
\draw [->,very thick,red,opacity=.5]
(c) to[bend left] (n1) (n1) -| (n2);
\end{tikzpicture}

\begin{tikzpicture}[remember picture]
\node (c) [circle,draw] {Big circle};
\draw [overlay,->,very thick,red,opacity=.5]
(c) to[bend left] (n1) (n1) -| (n2);
\end{tikzpicture}

\begin{figure}[h]
\centering
\begin{minipage}{0.4\linewidth}
\begin{tikzpicture}[remember picture]
\node (c) [circle,draw] {Big circle};
\draw [overlay,->,very thick,red,opacity=.5]
(c) to[bend left] (n1) (n1) -| (n2);
\end{tikzpicture}
\end{minipage}
\begin{minipage}{0.5\linewidth}
\begin{verbatim}
\begin{tikzpicture}[remember picture]
\node (c) [circle,draw] {Big circle};
\draw [overlay,->,very thick,red,opacity=.5]
(c) to[bend left] (n1) (n1) -| (n2);
\end{tikzpicture}
\end{verbatim}
\end{minipage}
\end{figure}
\end{document}